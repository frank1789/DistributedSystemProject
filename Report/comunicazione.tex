\section{Comunicazione}
All'interno del sistema realizzato i robot sono in comunicazione tramite rete
\textsc{Wi-Fi} ideale senza degradazione di segnale e conseguente perdita di
informazione.
Il primo tipo di comunicazione è quella disponibile tra robot che permette in
un range limitato l'assegnazione e la coordinazione efficiente della
destinazione per lo svolgimento della missione.
La seconda tipologia di comunicazione riguarda la stima della posizione dei
robot mediante \emph{particle filter} per la localizzazione rispetto hotspots 
\textsc{Wi-Fi} intesi, in questo caso, come punti di riferimento virtuali.
Generati casualmente o tramite input dell'utente all'interno
della mappa.
