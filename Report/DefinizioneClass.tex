\subsection{Classe Map}
\label{ssec:ClassMap}
La classe Map gestisce la generazione di una nuova mappa di dimensioni variabili
o il caricamento di una precedentemente salvata. Nel costruttore è possibile
specificare come primo argomento se desidera una nuova mappa o una esistente
mediante stringa ``new" o ``load".
Fornendo come parametri in ingresso la dimensione della mappa desiderata.
Così si avvia il processo di generazione procedurale visto
nella sez. \ref{ssec:k-d}. Opzionalmente è possibile specificare se si
desiderano più punti di riferimento virtuali oltre quelli impostati di default.
Con la stringa ``load", invece è possibile richiamare una mappa esistente da
\textsc{gui}.
All'interno della classe sono salvati, come vettori, i punti che costituiscono
la mappa e le linee che li collegano. Questi sono interpretati dalla funziona
che simula il processo di scansione del \textsc{lidar}.
Per mappe di grandi dimensioni si estraggono i punti liberi che non siano
collegati a linee. Questo permette di fornire al robot le condizioni iniziali
all'avvio della simulazione evitando che venga posizionato all'interno di
qualche parete.

\subsection{Classe Robot}
\label{ssec:ClassRobot}
La classe Robot definita nel progetto permette di generare uno o più istanze di
questo oggetto permettendo così avere un comportamento univoco.
Tali istanze condividono le stesse proprietà non modificabili dall'esterno,
infatti sono messe a disposizione dell'utente metodi per l'accesso e la
modifica di queste.
Questo permette di avere robot che condividono, una volta configurati, gli
stessi parametri costanti i quali sono:
\begin{enumerate*}[label={\alph*)},font={\bfseries}]
	\item geometrici;
	\item risoluzione dei sensori;
	\item memoria di massa riservata per le scansioni;
	\item memoria di massa riservata per occupacy grid.
\end{enumerate*}
Tutti i paramenti salvati all'interno della classe sono accessibili in solo
lettura dall'esterno.
Infatti una volta create una o più istanze del robot queste sono posizionate
all'interno della mappa di cui non conoscono nulla se non quello che
percepiscono tramite \textsc{lidar} ogni 10 \si{\hertz}. Successivamente il
robot si muoverà in direzione delle coordinate obiettivo pianificando il
percorso di volta in volta secondo le informazioni ricevute dalla funzione
potenziali artificiali, vista precedentemente nella
sez. \ref{sec:soluzioneprop}.\ref{ssec:ArtPotField}.

\subsection{Classe Particle Filter}
\label{ssec:ClassPF}
La classe Particle Filter mette a disposizione due funzioni pubbliche:
\begin{enumerate*}[label={\alph*)},font={\bfseries}]
	\item il costruttore di classe;
	\item il metodo \emph{update}.
\end{enumerate*}
Il costruttore è utilizzato per generare
un'istanza di questo oggetto legata all'altra classe robot da cui riceve le
informazioni relative alle caratteristiche dei sensori e il vettore velocità.
Inoltre la classe riceve dall'altra classe Map, \ref{ssec:ClassMap}, le
posizioni dei punti di riferimento virtuali precedentemente generati.
Con il metodo \emph{update} invece viene aggiornata la stima della posizione del
robot ad ogni iterazione per la lunghezza della simulazione.
