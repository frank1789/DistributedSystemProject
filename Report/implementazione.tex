\section{Implementazione}
\label{sec:implementazione}
Per la realizzazione del progetto si è optato per un simulazione con la suite
\textsc{matlab}, realizzando un software basato su un'architettura a oggetti
e funzionale per i motivi indicati di seguito.
\subsection{Esecuzione}
Una volta generato lo scenario la simulazione inizia con il posizionamento di
uno o più robot all'interno della mappa come condizione iniziale e assegnato un
obiettivo da raggiungere.
Questi non conosco l'ambiente in cui si trovano e quindi tentano di raggiungere
l'obiettivo nella maniera più rapida impostando un traiettoria rettilinea
determinata come norma tra due punti.
La pianificazione della traiettoria viene modificata in base alle rilevazioni
effettuate dal \textsc{lidar} in tal modo essi evitano l'ostacolo perché repulsi.
Per la creazione delle matrici di occupazione locali vengono utilizzate le misure
fornite dal lidar, tali misure vengono riproiettate nel sistema di riferimento del robot
e registrate all'interno di una matrice utilizzando come peso per ogni misura l'equazione
fornita da \eqref{eq:lidargrid}.
Queste verranno integrate in una matrice globale che sarà aggiornata ogni secondo. Il
passaggio da matrice locale a matrice globale avviene rototraslando rispetto la posizione
stimata del robot fornita dal particle filter e registrata all'interno del robot.La matrice
globale del robot si assume essere già dimensionata in base alla mappa da esplorare.
Il robot durante la sua esplorazione è programmato per affiancare il primo muro visibile e
percorrerlo,nel caso in cui durante questa operazione si dovessero presentare zone già visitate,
esso cercherà di raggiungere punti non ancora esplorati all'interno della stanza, questa operazione
viene condotta registrando le zone già esplorate dal robot all'interno di una seconda matrice assunta
anch'essa di una dimensione tale da permettere di contenere la stanza con la risoluzione voluta.
Il robot quando risulterà essere nel raggio di comunicazione con altri robot avvierà il meccanismo di Utility,
in base al quale sarà assegnato ad ogni robot comunicante un nuovo target da raggiungere che massimizzi
l'utilità, calcolata con la equazione \eqref{eq:componet3}. Per evitare la ricomunicazione tra robot è stato
inserito un delay che deve trascorrere tra una comunicazione e l'altra prima che i 2 robot possano ricomunicare
tra loro.La comunicazione come già detto risulta essere ideale senza degradazione di segnale.
Durante la comunicazione i robot condivideranno la mappa contenente le zone già visitate in modo da
non permettere il passaggio in zone già viste da altri robot.
%
\input{introOOP}
\subsection{Classe Map}
\label{ssec:ClassMap}
La classe Map gestisce la generazione di una nuova mappa di dimensioni variabili
o il caricamento di una precedentemente salvata. Nel costruttore è possibile
specificare come primo argomento se desidera una nuova mappa o una esistente
mediante stringa ``new" o ``load".
Fornendo come parametri in ingresso la dimensione della mappa desiderata.
Così si avvia il processo di generazione procedurale visto
nella sez. \ref{ssec:k-d}. Opzionalmente è possibile specificare se si
desiderano più punti di riferimento virtuali oltre quelli impostati di default.
Con la stringa ``load", invece è possibile richiamare una mappa esistente da
\textsc{gui}.
All'interno della classe sono salvati, come vettori, i punti che costituiscono
la mappa e le linee che li collegano. Questi sono interpretati dalla funziona
che simula il processo di scansione del \textsc{lidar}.
Per mappe di grandi dimensioni si estraggono i punti liberi che non siano
collegati a linee. Questo permette di fornire al robot le condizioni iniziali
all'avvio della simulazione evitando che venga posizionato all'interno di
qualche parete.

\subsection{Classe Robot}
\label{ssec:ClassRobot}
La classe Robot definita nel progetto permette di generare uno o più istanze di
questo oggetto permettendo così avere un comportamento univoco.
Tali istanze condividono le stesse proprietà non modificabili dall'esterno,
infatti sono messe a disposizione dell'utente metodi per l'accesso e la
modifica di queste.
Questo permette di avere robot che condividono, una volta configurati, gli
stessi parametri costanti i quali sono:
\begin{enumerate*}[label={\alph*)},font={\bfseries}]
	\item geometrici;
	\item risoluzione dei sensori;
	\item memoria di massa riservata per le scansioni;
	\item memoria di massa riservata per occupacy grid.
\end{enumerate*}
Tutti i paramenti salvati all'interno della classe sono accessibili in solo
lettura dall'esterno.
Infatti una volta create una o più istanze del robot queste sono posizionate
all'interno della mappa di cui non conoscono nulla se non quello che
percepiscono tramite \textsc{lidar} ogni 10 \si{\hertz}. Successivamente il
robot si muoverà in direzione delle coordinate obiettivo pianificando il
percorso di volta in volta secondo le informazioni ricevute dalla funzione
potenziali artificiali, vista precedentemente nella
sez. \ref{sec:soluzioneprop}.\ref{ssec:ArtPotField}.

\subsection{Classe Particle Filter}
\label{ssec:ClassPF}
La classe Particle Filter mette a disposizione due funzioni pubbliche:
\begin{enumerate*}[label={\alph*)},font={\bfseries}]
	\item il costruttore di classe;
	\item il metodo \emph{update}.
\end{enumerate*}
Il costruttore è utilizzato per generare
un'istanza di questo oggetto legata all'altra classe robot da cui riceve le
informazioni relative alle caratteristiche dei sensori e il vettore velocità.
Inoltre la classe riceve dall'altra classe Map, \ref{ssec:ClassMap}, le
posizioni dei punti di riferimento virtuali precedentemente generati.
Con il metodo \emph{update} invece viene aggiornata la stima della posizione del
robot ad ogni iterazione per la lunghezza della simulazione.

