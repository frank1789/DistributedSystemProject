Per la creazione delle matrici di occupazione locali vengono utilizzate le misure 
fornite dal lidar, tali misure vengono riproiettate nel sistema di riferimento del robot 
e registrate all'interno di una matrice utilizzando come peso per ogni misura l'equazione
fornita da (\ref{eq:lidargrid}.
Queste verranno integrate in una matrice globale che sarà aggiornata ogni secondo. Il
passaggio da matrice locale a matrice globale avviene rototraslando rispetto la posizione
stimata del robot fornita dal particle filter e registrata all'interno del robot.La matrice 
globale del robot si assume essere già dimensionata in base alla mappa da esplorare.

Il robot durante la sua esplorazione è programmato per affiancare il primo muro visibile e 
percorrerlo,nel caso in cui durante questa operazione si dovessero presentare zone già visitate,
esso cercherà di raggiungere punti non ancora esplorati all'interno della stanza, questa operazione
viene condotta registrando le zone già esplorate dal robot all'interno di una seconda matrice assunta
anch'essa di una dimensione tale da permettere di contenere la stanza con la risoluzione voluta.


Il robot quando risulterà essere nel raggio di comunicazione con altri robot avvierà il meccanismo di Utility,
in base al quale sarà assegnato ad ogni robot comunicante un nuovo target da raggiungere che massimizzi 
l'utilità, calcolata con la equazione (\ref{eq:componet3}). Per evitare la ricomunicazione tra robot è stato 
inserito un delay che deve trascorrere tra una comunicazione e l'altra prima che i 2 robot possano ricomunicare 
tra loro.La comunicazione come già detto risulta essere ideale senza degradazione di segnale.

Durante la comunicazione i robot condivideranno la mappa contenente le zone già visitate in modo da
non permettere il passaggio in zone già viste da altri robot.