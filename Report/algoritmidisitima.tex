\subsection{Algoritmi di stima}
Il sistema robotico preso in esame presenta degli errori all'ingresso degli 
encoder, tale errore porta il sistema a manifestare nel lungo periodo il noto 
comportamento di \emph{drift}. Nasce allora il problema di correggere e stimare
la posizione in base ad un dato modello dello stato presentato già 
precedentemente il modello odometrico e un modello di sensore esterno al 
robot che fornisca una misura approssimata della sua posizione. I metodi 
presentati saranno 2 , un primo di semplice applicazione ipotizza il 
ricevimento da parte del robot di un segnale GPS che viene integrato e 
sfruttato tramite il filtro di Kalman, un'altro modello e caso di studio sarà 
l'utilizzo di torrette Wi-fi integrate all'interno dell'ambiente che si vuole 
esplorare , in questo caso sarà invece implementato il particle filter. 
Una breve ricapitolazione del filtro di kalman sarà presentata mentre per il 
particle filter un'analisi più approfondita sarà condotta in quanto non 
argomento del corso.
%
\subsubsection{Filtro di Kalman}
Un filtro di Kalman è uno stimatore ottimale, ovvero deduce i parametri di 
interesse da osservazioni indirette, imprecise e incerte. È ricorsivo, così 
nuove misure possono essere elaborate al loro arrivo. 
Se il rumore è gaussiano, il filtro di Kalman minimizza l'errore quadratico 
medio dei parametri stimati.
Dato solo la media e la deviazione standard del rumore, il filtro di Kalman è 
il miglior stimatore lineare. Gli stimatori non lineari possono essere migliori.
%
\subsubsection{Filtro Particellare}
%
L'algoritmo di localizzazione PF procede come segue: si inizializzano $n$ 
particelle in una mappa.
Ogni particella è un vettore di stato 3 per 1 del veicolo ad ognuna di queste 
si applica il modello di plant e se ne aggiunge un rumore al vettore di controllo $u$. 
Successivamente per ogni particella se ne prevede l'osservazione e si confronta 
questo con il valore misurato, tale confronto porterà al calcolo 
dell'innovazione o di ciò che definiremo peso della particella.
Si selezionano le particelle che meglio spiegano l'osservazione, un modo per 
farlo è quello di costruire una pdf che descriva i campioni e i loro pesi, e 
poi riselezionare un nuovo set di particelle da questa pdf.
La stima della posizione del robot fornita dal filtro e la media di questo 
nuovo ricampionamento. %TODO completare? 
%
Il punto cruciale è che non richiede alcuna ipotesi di linearizzazione (non ci 
sono jacobiani coinvolti) e non ci sono ipotesi Gaussiane. È particolarmente 
adatto ai problemi con piccoli spazi di stato mentre in caso di vettori di 
stato grandi diventa computazionalmente pesante.
