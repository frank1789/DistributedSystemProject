%----------------------------------------------------------------------------------------
%	PACKAGES AND OTHER DOCUMENT CONFIGURATIONS
%----------------------------------------------------------------------------------------
\documentclass[11pt,a4paper,italian,twoside,twocolumn]{article}
\pagenumbering{arabic}
\usepackage{setspace}
%\onehalfspace
\usepackage[hmarginratio=1:1,top=32mm,columnsep=20pt]{geometry}
\usepackage{fancyhdr}
\usepackage{multirow}
\usepackage{multicol}
\usepackage[section]{placeins}
%--Text and Style------------------------------------------------------------------------
\usepackage{microtype} 			% Slightly tweak font spacing for aesthetics
\usepackage[italian]{babel} 		% Language hyphenation and typographical rules
\usepackage[T1]{fontenc}
\usepackage[utf8]{inputenc}
\usepackage{ae}
\usepackage{relsize}
\usepackage{csquotes} 
\usepackage{amsmath}
\usepackage{amsfonts}
\usepackage{mathdots}
\usepackage{mathtools}
\usepackage[colorlinks=true]{hyperref}
\hypersetup{
	bookmarksnumbered=true,
	linkcolor=black,
	citecolor=black,
	%pagecolor=black,
	urlcolor=black,
}
\usepackage{verbatim}
\usepackage{alltt}
\DeclareMathOperator{\sgn}{sgn}
\DeclareMathOperator{\RealNumber}{\rm I\!R}
\DeclarePairedDelimiter{\abs}{\lvert}{\rvert}
\DeclarePairedDelimiter{\norma}{\lVert}{\rVert}
\usepackage{lipsum}
\usepackage{siunitx}
\usepackage[inline]{enumitem}
\usepackage{lettrine}

%---Figure------------------------------------------------------------------------------

\usepackage{graphicx}
\graphicspath{{./imgs/}}
\renewcommand{\figurename}{Fig.}
\usepackage{subfig}

%---------------------------------------------------------------------------------------
\usepackage{algorithmicx}
\usepackage[ruled]{algorithm}
\usepackage{algpseudocode}
\usepackage{listings}
\usepackage{xcolor}
\definecolor{halfgray}{gray}{0.55}
\definecolor{ipython_frame}{RGB}{207, 207, 207}
\definecolor{ipython_bg}{RGB}{247, 247, 247}
\definecolor{ipython_red}{RGB}{186, 33, 33}
\definecolor{ipython_green}{RGB}{0, 128, 0}
\definecolor{ipython_cyan}{RGB}{64, 128, 128}
\definecolor{ipython_purple}{RGB}{170, 34, 255}
\lstset{
    breaklines=true,
    %
    extendedchars=true,
    literate=
    {á}{{\'a}}1 {é}{{\'e}}1 {í}{{\'i}}1 {ó}{{\'o}}1 {ú}{{\'u}}1
    {Á}{{\'A}}1 {É}{{\'E}}1 {Í}{{\'I}}1 {Ó}{{\'O}}1 {Ú}{{\'U}}1
    {à}{{\`a}}1 {è}{{\`e}}1 {ì}{{\`i}}1 {ò}{{\`o}}1 {ù}{{\`u}}1
    {À}{{\`A}}1 {È}{{\'E}}1 {Ì}{{\`I}}1 {Ò}{{\`O}}1 {Ù}{{\`U}}1
    {ä}{{\"a}}1 {ë}{{\"e}}1 {ï}{{\"i}}1 {ö}{{\"o}}1 {ü}{{\"u}}1
    {Ä}{{\"A}}1 {Ë}{{\"E}}1 {Ï}{{\"I}}1 {Ö}{{\"O}}1 {Ü}{{\"U}}1
    {â}{{\^a}}1 {ê}{{\^e}}1 {î}{{\^i}}1 {ô}{{\^o}}1 {û}{{\^u}}1
    {Â}{{\^A}}1 {Ê}{{\^E}}1 {Î}{{\^I}}1 {Ô}{{\^O}}1 {Û}{{\^U}}1
    {œ}{{\oe}}1 {Œ}{{\OE}}1 {æ}{{\ae}}1 {Æ}{{\AE}}1 {ß}{{\ss}}1
    {ç}{{\c c}}1 {Ç}{{\c C}}1 {ø}{{\o}}1 {å}{{\r a}}1 {Å}{{\r A}}1
    {€}{{\EUR}}1 {£}{{\pounds}}1
}

%%
%% Python definition (c) 1998 Michael Weber
%% Additional definitions (2013) Alexis Dimitriadis
%% modified by me (should not have empty lines)
%%
\lstdefinelanguage{iPython}{
    morekeywords={access,and,break,class,continue,def,del,elif,else,except,exec,finally,for,from,global,if,import,in,is,lambda,not,or,pass,print,raise,return,try,while},%
    %
    % Built-ins
    morekeywords=[2]{abs,all,any,basestring,bin,bool,bytearray,callable,chr,classmethod,cmp,compile,complex,delattr,dict,dir,divmod,enumerate,eval,execfile,file,filter,float,format,frozenset,getattr,globals,hasattr,hash,help,hex,id,input,int,isinstance,issubclass,iter,len,list,locals,long,map,max,memoryview,min,next,object,oct,open,ord,pow,property,range,raw_input,reduce,reload,repr,reversed,round,set,setattr,slice,sorted,staticmethod,str,sum,super,tuple,type,unichr,unicode,vars,xrange,zip,apply,buffer,coerce,intern},%
    %
    sensitive=true,%
    morecomment=[l]\#,%
    morestring=[b]',%
    morestring=[b]",%
    %
    morestring=[s]{'''}{'''},% used for documentation text (mulitiline strings)
    morestring=[s]{"""}{"""},% added by Philipp Matthias Hahn
    %
    morestring=[s]{r'}{'},% `raw' strings
    morestring=[s]{r"}{"},%
    morestring=[s]{r'''}{'''},%
    morestring=[s]{r"""}{"""},%
    morestring=[s]{u'}{'},% unicode strings
    morestring=[s]{u"}{"},%
    morestring=[s]{u'''}{'''},%
    morestring=[s]{u"""}{"""},%
    %
    % {replace}{replacement}{lenght of replace}
    % *{-}{-}{1} will not replace in comments and so on
    literate=
    *{+}{{{\color{ipython_purple}+}}}1
    {-}{{{\color{ipython_purple}-}}}1
    {*}{{{\color{ipython_purple}$^\ast$}}}1
    {/}{{{\color{ipython_purple}/}}}1
    {^}{{{\color{ipython_purple}\^{}}}}1
    {?}{{{\color{ipython_purple}?}}}1
    {!}{{{\color{ipython_purple}!}}}1
    {\%}{{{\color{ipython_purple}\%}}}1
    {<}{{{\color{ipython_purple}<}}}1
    {>}{{{\color{ipython_purple}>}}}1
    {|}{{{\color{ipython_purple}|}}}1
    {\&}{{{\color{ipython_purple}\&}}}1
    {~}{{{\color{ipython_purple}~}}}1
    %
    {==}{{{\color{ipython_purple}==}}}2
    {<=}{{{\color{ipython_purple}<=}}}2
    {>=}{{{\color{ipython_purple}>=}}}2
    %
    {+=}{{{+=}}}2
    {-=}{{{-=}}}2
    {*=}{{{$^\ast$=}}}2
    {/=}{{{/=}}}2,
    %
    literate=
    {á}{{\'a}}1 {é}{{\'e}}1 {í}{{\'i}}1 {ó}{{\'o}}1 {ú}{{\'u}}1
    {Á}{{\'A}}1 {É}{{\'E}}1 {Í}{{\'I}}1 {Ó}{{\'O}}1 {Ú}{{\'U}}1
    {à}{{\`a}}1 {è}{{\`e}}1 {ì}{{\`i}}1 {ò}{{\`o}}1 {ù}{{\`u}}1
    {À}{{\`A}}1 {È}{{\'E}}1 {Ì}{{\`I}}1 {Ò}{{\`O}}1 {Ù}{{\`U}}1
    {ä}{{\"a}}1 {ë}{{\"e}}1 {ï}{{\"i}}1 {ö}{{\"o}}1 {ü}{{\"u}}1
    {Ä}{{\"A}}1 {Ë}{{\"E}}1 {Ï}{{\"I}}1 {Ö}{{\"O}}1 {Ü}{{\"U}}1
    {â}{{\^a}}1 {ê}{{\^e}}1 {î}{{\^i}}1 {ô}{{\^o}}1 {û}{{\^u}}1
    {Â}{{\^A}}1 {Ê}{{\^E}}1 {Î}{{\^I}}1 {Ô}{{\^O}}1 {Û}{{\^U}}1
    {œ}{{\oe}}1 {Œ}{{\OE}}1 {æ}{{\ae}}1 {Æ}{{\AE}}1 {ß}{{\ss}}1
    {ç}{{\c c}}1 {Ç}{{\c C}}1 {ø}{{\o}}1 {å}{{\r a}}1 {Å}{{\r A}}1
    {€}{{\EUR}}1 {£}{{\pounds}}1,
    %
%   identifierstyle=\color{red}\ttfamily,
    commentstyle=\color{ipython_cyan}\ttfamily,
    stringstyle=\color{ipython_red}\ttfamily,
    keepspaces=true,
    showspaces=false,
    showstringspaces=false,
    %
    rulecolor=\color{ipython_frame},
    frame=single,
    frameround={t}{t}{t}{t},
    framexleftmargin=6mm,
    numbers=left,
    numberstyle=\tiny\color{halfgray},
    %
    %
    backgroundcolor=\color{ipython_bg},
    %   extendedchars=true,
    basicstyle=\scriptsize\ttfamily,
    keywordstyle=\color{ipython_green}\ttfamily,
}

%------Matlab scheme color-code----------------------------------
\usepackage{matlab-prettifier}
\lstset{
	style = Matlab-editor,
	basicstyle=\mlttfamily,
    %Style frame and number
    rulecolor = \color{ipython_frame},
    frame	=single,
    frameround={t}{t}{t}{t},
    framexleftmargin=6mm,
    numbers=left,
    numberstyle=\tiny\color{halfgray},
	%background color frame
    backgroundcolor=\color{ipython_bg}   
}

%---Table------------------------------------------------------------
\usepackage{tabularx}
\usepackage{array}
\usepackage{color}
\usepackage{adjustbox}
\usepackage{colortbl}
\usepackage{pgfplotstable}
\usepackage{makecell}
\usepackage{booktabs}

%--------------------------------------------------------------------
% Allows abstract customization
\usepackage{abstract}
% Set the "Abstract" text to bold
\renewcommand{\abstractnamefont}{\normalfont\bfseries} 
% Set the abstract itself to small italic text
\renewcommand{\abstracttextfont}{\normalfont\small\itshape} 
% Allows customization of titles
\usepackage{titlesec} 
% Roman numerals for the sections
\renewcommand\thesection{\Roman{section}} 
% roman numerals for subsections
\renewcommand\thesubsection{\roman{subsection}} 
% Change the look of the section titles
\titleformat{\section}[block]{\large\scshape\centering}{\thesection.}{1em}{} 
% Change the look of the section titles
\titleformat{\subsection}[block]{\large}{\thesubsection.}{1em}{} 
\usepackage{titling} % Customizing the title section

%----------------------------------------------------------------------------------------
\usepackage{tikz,fp,ifthen,fullpage}
\usepackage{pgfmath, pgfplots, xparse}
\usepgfplotslibrary{fillbetween}
\usetikzlibrary{backgrounds, arrows}
\usetikzlibrary{decorations.pathmorphing,fit,through}
\usetikzlibrary{shapes,decorations,shadows}
\usetikzlibrary{fadings,patterns,mindmap}
\usepackage{tikz-dimline, calc}
\pgfplotsset{compat=newest}

%----------------------------------------------------------------------------------------
%	TITLE SECTION
%----------------------------------------------------------------------------------------
\setlength{\droptitle}{-4\baselineskip} % Move the title up
\pretitle{\begin{center}\Huge\bfseries} % Article title formatting
\posttitle{\end{center}} % Article title closing formatting
\title{TITOLO DA INSERIRE} % Article title %TODO scegliere e inserire titolo
\author{%
	\textsc{Francesco Argentieri}\thanks{ID: 183892}\\[1ex] % Your name
	\normalsize Università di Trento \\ % Your institution
	% Your email address
	\normalsize \href{mailto:francesco.argentieri@studenti.unitn.it}{francesco.argentieri@studenti.unitn.it}%
%% Uncomment if 2 authors are required, duplicate these 4 lines if more
\and 
\textsc{Giacomo Mazzaglia}\thanks{ID: 123456} \\[1ex] % Second author's name %TODO matricola Giacomo
\normalsize Università di Trento \\ % Second author's institution
\normalsize \href{mailto:giacomo.mazzaglia@studenti.unitn.it}{giacomo.mazzaglia@studenti.unitn.it}%
}
%\date{} % Leave empty to omit a date
\renewcommand{\maketitlehookd}{\begin{abstract}
In questo lavoro, consideriamo il problema di esplorare un ambiente sconosciuto con un team di robot. Come nell'esplorazione di robot singoli, l'obiettivo è di ridurre al minimo il tempo di esplorazione complessivo. Il problema chiave da risolvere nel contesto di robot multipli è quello di scegliere i punti di destinazione appropriati per i singoli robot in modo che possano esplorare contemporaneamente diverse regioni dell'ambiente. Presentiamo un approccio per il coordinamento di più robot, che tiene conto simultaneamente del costo di raggiungere un punto target e della sua utilità. Ogni volta che un punto target viene assegnato a un robot specifico, l'utilità dell'area inesplorata visibile da questa posizione target viene ridotta. In questo modo, le diverse posizioni di destinazione vengono assegnate ai singoli robot. Descriviamo inoltre come il nostro algoritmo può essere esteso a situazioni in cui il raggio di comunicazione dei robot è limitato.Per la stima delle posizioni dei robot è stato utilizzato il filtro particellare,assumendo una comunicazione con delle ancore wi-fi. I risultati dimostrano che la nostra tecnica distribuisce efficacemente i robot sull'ambiente e consente loro di compiere rapidamente la loro missione.
\end{abstract}
} %TODO modificare abstract nel file
%----------------------------------------------------------------------------------------
\begin{document}
	% Print the title
	\maketitle	
	%Other chapter
	\clearpage
	\newpage
	\section{Modello del sistema}
\subsection{Modello cinematico}
Il robot è basato sul modello dell'uniciclo a trazione differenziale, la configurazione è completamente descritta da $\mathbf{q} = [x \, y \, \theta]^T$, dove $(x,y)$ sono le coordinate cartesiane del punto di contatto con il suolo e $\theta$ è l'orientamento della ruota rispetto l'asse $x$.\cite{siciliano2008robotica}, come in figura \ref{fig:model}.
Il modello cinematico dell'uniciclo è descritto dall'equazioni (\ref{eq:modelcinematico}):
\begin{equation}
\label{eq:modelcinematico}
	\begin{bmatrix}
		\dot{x} \\ 
		\dot{y} \\ 
		\dot{\theta}
	\end{bmatrix} = 
	\begin{bmatrix}
		\cos \theta \\
		\sin \theta \\
		0
	\end{bmatrix} \, v + 
	\begin{bmatrix}
		0 \\
		0 \\
		1
	\end{bmatrix} \, \omega
\end{equation}
\begin{table}[htb]
	\centering
	\caption{Riepilogo dimensioni}
	\label{tab:dimensrobot}
	\begin{tabular}{lcS[table-format=3.2]}
	\toprule
	\multicolumn{3}{c}{dimensioni}\\
	\midrule
      raggio ruote  & [\si{\metre}] & 0.07\\ % dimension wheel [m]
      interasse     & [\si{\metre}] & 0.30\\ % dimension interaxle [m]
     \bottomrule
\end{tabular}
\end{table}
Il robot ha le dimensioni riportate in tabella \ref{tab:dimensrobot}.
Questo è equipaggiato con un sensore virtuale \emph{lidar}, basato sul modello Hokuyo URG-04LX, collocato al centro della struttura in modo tale da evitare errori di offset, di seguito se ne riportano le caratteristiche, di cui adattate ad hoc per la simulazione. Come sensori propriocettivi presenta due encoder incrementali virtuali calettati sull'asse delle ruote, le caratteristiche di entrambi sono riportate in tabella \ref{tab:sensordata}. Una rappresentazione del robot è osservabile in figura \ref{fig:model}.
\begin{table}[htb]
	\centering
	\caption{Specifiche sensori}
	\label{tab:sensordata}
	\begin{tabular}{lcS[table-format=3.2]}
	\toprule
	\multicolumn{3}{c}{specifiche lidar virtuale}\\
	\midrule
 		risoluzione angolare & [\si{\degree}]	& 0.36\\  % [deg] laser sensor parameters\\
 		angolo di scansione  & [\si{\degree}]	& 180.00\\
 		massima distanza		 & [\si{\metre}]		& 4.00	\\ % [m] laser sensor parameters Max FOV
 		minima distanza 		 & [\si{\metre}]		& 0.02	\\ % [m] laser sensor parameters min FOV
 		risoluzione 			 & [\si{\milli\metre}]& 1.00\\
%        % noise
%        laser_rho_sigma     = 0.02;             % variance
%        laser_theta_sigma   = 0.1 * (pi / 180); % variance
	\bottomrule
	\multicolumn{3}{c}{specifiche encoder virtuale}\\
	\midrule
	 risoluzione &  $2 \cdot (\frac{\pi}{2600})$\\   % encoder quantization
        % encoder noise
        %enc_mu = 0;                           % mean
        %enc_sigma = 2 * (2 * pi / 2600) / 3;  % variance
	\bottomrule
	\end{tabular}
\end{table}
%\clearpage
%\onecolumn
\begin{figure}[!h]
\centering
    \resizebox{.8\linewidth}{!}{\begin{tikzpicture} [>=latex]
% \draw [help lines] (0,0) grid (8, 8);
% \foreach \x in {0,1,...,8}
%   \draw [help lines] (\x,0) node [below,%
%          font=\footnotesize] {$\x$} -- (\x,0);
%\foreach \y in {0,1,...,8}
%   \draw [help lines] (0,\y) node [left,%
%          font=\footnotesize] {$\y$} -- (0,\y);
%body robot
 \draw [fill=lightgray, fill opacity=0.5](4, 4) circle (2.25);
 \def\drawwheel{
 \draw [rounded corners=15,fill=lightgray, pattern color=gray] (0.5, 2.5)  rectangle (1.5, 5.5);
 \draw [rounded corners=15,fill=lightgray, pattern color=gray] (6.5, 2.5) rectangle (7.5, 5.5);
 \draw (1.5, 4) -- (6.5, 4);}
 % quote wheel
 \dimline  [color=gray, 
                 %line style={thick},
                %extension start style={gray,thin},
                %extension end style={gray,thin},
               extension start length=1cm,
              extension end length=1cm,
                ]{(0, 4)}{ (0, 5.5)}{$r$};
 % quote track
 \dimline    [color=gray,
                % line style={thick},
                %extension start style={gray,thin},
                %extension end style={gray,thin},
                extension start length=-1cm,
                extension end length=-1cm
                ]{(1, 1.25)}{ (7, 1.25)}{$b$};
 % lidar
  	\draw [fill=black](3.5,3.5) rectangle (4.5,4.5);
  	\node at (4,3.5) [below]{\tiny \textsc{lidar}};
  	\draw [fill=black](3.75,4.5) rectangle (4.25,4.65);
  	\draw [color=green, fill=green!25, fill opacity=0.5](-4,4.65) -- (12,4.65) arc(0:180:8) --cycle;
  	\draw [color=green] (4,4.65) -- +(39:8);
  % encoder
  \draw [fill=black] (1.90,3.80) rectangle (2.10,4.20);
  \node at (2,3.80) [below]{\tiny \textsc{encoder}};
  \draw [fill=black] (5.90,3.80) rectangle (6.10,4.20);
  \node at (6,3.80) [below]{\tiny \textsc{encoder}};
   % vector
 	\draw [->, blue] (4, 4) -- (4, 8) node[left]{$v$};
 	%\draw [->, red] (1, 4) -- (1,7) node[left]{$\omega_{r}$};
 	%\draw [->, red] (7, 4) -- (7,7)	 node[left]{$\omega_{l}$};
	\draw [->, red] (5.5,4) arc (0:(165):1.5) node[below]{$\omega$};
	\drawwheel;
	% Reference system 0
 	\coordinate [label = below: \scriptsize $RF0$] (A) at (0,0);
 	\coordinate	(Bx) at	($(A)+1.5*(0:1)$);
 	\coordinate	(By)	 at	($(A)+1.5*(90:1)$);
 	\draw [->] 	(A) -- (By) node[left]{$y$};
 	\draw [->] 	(A) -- (Bx) node[above]{$x$};
 \end{tikzpicture}}
\caption{modello cinematico}
\label{fig:model}
\end{figure}
%\twocolumn
%nuova sezione


	\newpage
	\section{Implementazione}
\label{sec:implementazione}
Per la realizzazione del progetto si è optato per una simulazione con la suite
\textsc{matlab}, realizzando un software basato su un'architettura a oggetti
e funzionale per i motivi indicati di seguito.
\subsection{Esecuzione}
Una volta generato lo scenario la simulazione inizia con il posizionamento di
uno o più robot all'interno della mappa come condizione iniziale e assegnato un
obiettivo da raggiungere.
Questi non conosco l'ambiente in cui si trovano e quindi tentano di raggiungere
l'obiettivo nella maniera più rapida impostando un traiettoria rettilinea
determinata come norma tra due punti.
La pianificazione della traiettoria viene modificata in base alle rilevazioni
effettuate dal \textsc{lidar} in tal modo essi evitano l'ostacolo perché repulsi.
Per la creazione delle matrici di occupazione locali vengono utilizzate le misure
fornite dal lidar, tali misure vengono riproiettate nel sistema di riferimento del robot
e registrate all'interno di una matrice utilizzando come peso per ogni misura l'equazione
fornita da \eqref{eq:lidargrid}.
Queste verranno integrate in una matrice globale che sarà aggiornata ogni secondo. Il
passaggio da matrice locale a matrice globale avviene rototraslando rispetto la posizione
stimata del robot fornita dal particle filter e registrata all'interno del robot.La matrice
globale del robot si assume essere già dimensionata in base alla mappa da esplorare.
Il robot durante la sua esplorazione è programmato per affiancare il primo muro visibile e
percorrerlo,nel caso in cui durante questa operazione si dovessero presentare zone già visitate,
esso cercherà di raggiungere punti non ancora esplorati all'interno della stanza, questa operazione
viene condotta registrando le zone già esplorate dal robot all'interno di una seconda matrice assunta
anch'essa di una dimensione tale da permettere di contenere la stanza con la risoluzione voluta.
Il robot quando risulterà essere nel raggio di comunicazione con altri robot avvierà il meccanismo di Utility,
in base al quale sarà assegnato ad ogni robot comunicante un nuovo target da raggiungere che massimizzi
l'utilità, calcolata con la equazione \eqref{eq:componet3}. Per evitare la ricomunicazione tra robot è stato
inserito un delay che deve trascorrere tra una comunicazione e l'altra prima che i 2 robot possano ricomunicare
tra loro.La comunicazione come già detto risulta essere ideale senza degradazione di segnale.
Durante la comunicazione i robot condivideranno la mappa contenente le zone già visitate in modo da
non permettere il passaggio in zone già viste da altri robot.
%
\input{introOOP}
\subsection{Classe Map}
\label{ssec:ClassMap}
La classe Map gestisce la generazione di una nuova mappa di dimensioni variabili
o il caricamento di una precedentemente salvata. Nel costruttore è possibile
specificare come primo argomento se desidera una nuova mappa o una esistente
mediante stringa ``new" o ``load".
Fornendo come parametri in ingresso la dimensione della mappa desiderata.
Così si avvia il processo di generazione procedurale visto
nella sez. \ref{ssec:k-d}. Opzionalmente è possibile specificare se si
desiderano più punti di riferimento virtuali oltre quelli impostati di default.
Con la stringa ``load", invece è possibile richiamare una mappa esistente da
\textsc{gui}.
All'interno della classe sono salvati, come vettori, i punti che costituiscono
la mappa e le linee che li collegano. Questi sono interpretati dalla funziona
che simula il processo di scansione del \textsc{lidar}.
Per mappe di grandi dimensioni si estraggono i punti liberi che non siano
collegati a linee. Questo permette di fornire al robot le condizioni iniziali
all'avvio della simulazione evitando che venga posizionato all'interno di
qualche parete.

\subsection{Classe Robot}
\label{ssec:ClassRobot}
La classe Robot definita nel progetto permette di generare uno o più istanze di
questo oggetto permettendo così avere un comportamento univoco.
Tali istanze condividono le stesse proprietà non modificabili dall'esterno,
infatti sono messe a disposizione dell'utente metodi per l'accesso e la
modifica di queste.
Questo permette di avere robot che condividono, una volta configurati, gli
stessi parametri costanti i quali sono:
\begin{enumerate*}[label={\alph*)},font={\bfseries}]
	\item geometrici;
	\item risoluzione dei sensori;
	\item memoria di massa riservata per le scansioni;
	\item memoria di massa riservata per occupacy grid.
\end{enumerate*}
Tutti i paramenti salvati all'interno della classe sono accessibili in solo
lettura dall'esterno.
Infatti una volta create una o più istanze del robot queste sono posizionate
all'interno della mappa di cui non conoscono nulla se non quello che
percepiscono tramite \textsc{lidar} ogni 10 \si{\hertz}. Successivamente il
robot si muoverà in direzione delle coordinate obiettivo pianificando il
percorso di volta in volta secondo le informazioni ricevute dalla funzione
potenziali artificiali, vista precedentemente nella
sez. \ref{sec:soluzioneprop}.\ref{ssec:ArtPotField}.

\subsection{Classe Particle Filter}
\label{ssec:ClassPF}
La classe Particle Filter mette a disposizione due funzioni pubbliche:
\begin{enumerate*}[label={\alph*)},font={\bfseries}]
	\item il costruttore di classe;
	\item il metodo \emph{update}.
\end{enumerate*}
Il costruttore è utilizzato per generare
un'istanza di questo oggetto legata all'altra classe robot da cui riceve le
informazioni relative alle caratteristiche dei sensori e il vettore velocità.
Inoltre la classe riceve dall'altra classe Map, \ref{ssec:ClassMap}, le
posizioni dei punti di riferimento virtuali precedentemente generati.
Con il metodo \emph{update} invece viene aggiornata la stima della posizione del
robot ad ogni iterazione per la lunghezza della simulazione.


%	\input{Result}
\newpage
%----------------------------------------------------------------------------------------
%	Bibliography
%----------------------------------------------------------------------------------------
	\clearpage
	\bibliographystyle{IEEEtran}
	\bibliography{bibliografia.bib}
\end{document}