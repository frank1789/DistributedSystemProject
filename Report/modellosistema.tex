\section{Modello del sistema}
\subsection{Modello cinematico}
Il robot è basato sul modello dell'uniciclo a trazione differenziale, la configurazione è completamente descritta da $\mathbf{q} = [x \, y \, \theta]^T$, dove $(x,y)$ sono le coordinate cartesiane del punto di contatto con il suolo e $\theta$ è l'orientamento della ruota rispetto l'asse $x$.\cite{siciliano2008robotica}, come in figura \ref{fig:model}.
Il modello cinematico dell'uniciclo è descritto dall'equazioni (\ref{eq:modelcinematico}):
\begin{equation}
\label{eq:modelcinematico}
	\begin{bmatrix}
		\dot{x} \\ 
		\dot{y} \\ 
		\dot{\theta}
	\end{bmatrix} = 
	\begin{bmatrix}
		\cos \theta \\
		\sin \theta \\
		0
	\end{bmatrix} \, v + 
	\begin{bmatrix}
		0 \\
		0 \\
		1
	\end{bmatrix} \, \omega
\end{equation}
\begin{table}[htb]
	\centering
	\caption{Riepilogo dimensioni}
	\label{tab:dimensrobot}
	\begin{tabular}{lcS[table-format=3.2]}
	\toprule
	\multicolumn{3}{c}{dimensioni}\\
	\midrule
      raggio ruote  & [\si{\metre}] & 0.07\\ % dimension wheel [m]
      interasse     & [\si{\metre}] & 0.30\\ % dimension interaxle [m]
     \bottomrule
\end{tabular}
\end{table}
Il robot ha le dimensioni riportate in tabella \ref{tab:dimensrobot}.
Questo è equipaggiato con un sensore virtuale \emph{lidar}, basato sul modello Hokuyo URG-04LX, collocato al centro della struttura in modo tale da evitare errori di offset, di seguito se ne riportano le caratteristiche, di cui adattate ad hoc per la simulazione. Come sensori propriocettivi presenta due encoder incrementali virtuali calettati sull'asse delle ruote, le caratteristiche di entrambi sono riportate in tabella \ref{tab:sensordata}. Una rappresentazione del robot è osservabile in figura \ref{fig:model}.
\begin{table}[htb]
	\centering
	\caption{Specifiche sensori}
	\label{tab:sensordata}
	\begin{tabular}{lcS[table-format=3.2]}
	\toprule
	\multicolumn{3}{c}{specifiche lidar virtuale}\\
	\midrule
 		risoluzione angolare & [\si{\degree}]	& 0.36\\  % [deg] laser sensor parameters\\
 		angolo di scansione  & [\si{\degree}]	& 180.00\\
 		massima distanza		 & [\si{\metre}]		& 4.00	\\ % [m] laser sensor parameters Max FOV
 		minima distanza 		 & [\si{\metre}]		& 0.02	\\ % [m] laser sensor parameters min FOV
 		risoluzione 			 & [\si{\milli\metre}]& 1.00\\
%        % noise
%        laser_rho_sigma     = 0.02;             % variance
%        laser_theta_sigma   = 0.1 * (pi / 180); % variance
	\bottomrule
	\multicolumn{3}{c}{specifiche encoder virtuale}\\
	\midrule
	 risoluzione &  $2 \cdot (\frac{\pi}{2600})$\\   % encoder quantization
        % encoder noise
        %enc_mu = 0;                           % mean
        %enc_sigma = 2 * (2 * pi / 2600) / 3;  % variance
	\bottomrule
	\end{tabular}
\end{table}
%\clearpage
%\onecolumn
\begin{figure}[!h]
\centering
    \resizebox{.8\linewidth}{!}{\begin{tikzpicture} [>=latex]
% \draw [help lines] (0,0) grid (8, 8);
% \foreach \x in {0,1,...,8}
%   \draw [help lines] (\x,0) node [below,%
%          font=\footnotesize] {$\x$} -- (\x,0);
%\foreach \y in {0,1,...,8}
%   \draw [help lines] (0,\y) node [left,%
%          font=\footnotesize] {$\y$} -- (0,\y);
%body robot
 \draw [fill=lightgray, fill opacity=0.5](4, 4) circle (2.25);
 \def\drawwheel{
 \draw [rounded corners=15,fill=lightgray, pattern color=gray] (0.5, 2.5)  rectangle (1.5, 5.5);
 \draw [rounded corners=15,fill=lightgray, pattern color=gray] (6.5, 2.5) rectangle (7.5, 5.5);
 \draw (1.5, 4) -- (6.5, 4);}
 % quote wheel
 \dimline  [color=gray, 
                 %line style={thick},
                %extension start style={gray,thin},
                %extension end style={gray,thin},
               extension start length=1cm,
              extension end length=1cm,
                ]{(0, 4)}{ (0, 5.5)}{$r$};
 % quote track
 \dimline    [color=gray,
                % line style={thick},
                %extension start style={gray,thin},
                %extension end style={gray,thin},
                extension start length=-1cm,
                extension end length=-1cm
                ]{(1, 1.25)}{ (7, 1.25)}{$b$};
 % lidar
  	\draw [fill=black](3.5,3.5) rectangle (4.5,4.5);
  	\node at (4,3.5) [below]{\tiny \textsc{lidar}};
  	\draw [fill=black](3.75,4.5) rectangle (4.25,4.65);
  	\draw [color=green, fill=green!25, fill opacity=0.5](-4,4.65) -- (12,4.65) arc(0:180:8) --cycle;
  	\draw [color=green] (4,4.65) -- +(39:8);
  % encoder
  \draw [fill=black] (1.90,3.80) rectangle (2.10,4.20);
  \node at (2,3.80) [below]{\tiny \textsc{encoder}};
  \draw [fill=black] (5.90,3.80) rectangle (6.10,4.20);
  \node at (6,3.80) [below]{\tiny \textsc{encoder}};
   % vector
 	\draw [->, blue] (4, 4) -- (4, 8) node[left]{$v$};
 	%\draw [->, red] (1, 4) -- (1,7) node[left]{$\omega_{r}$};
 	%\draw [->, red] (7, 4) -- (7,7)	 node[left]{$\omega_{l}$};
	\draw [->, red] (5.5,4) arc (0:(165):1.5) node[below]{$\omega$};
	\drawwheel;
	% Reference system 0
 	\coordinate [label = below: \scriptsize $RF0$] (A) at (0,0);
 	\coordinate	(Bx) at	($(A)+1.5*(0:1)$);
 	\coordinate	(By)	 at	($(A)+1.5*(90:1)$);
 	\draw [->] 	(A) -- (By) node[left]{$y$};
 	\draw [->] 	(A) -- (Bx) node[above]{$x$};
 \end{tikzpicture}}
\caption{modello cinematico}
\label{fig:model}
\end{figure}
%\twocolumn
%nuova sezione
\subsection{Pianificazione}
\label{ssec:ArtPotField}
La pianificazione del percorso per i robot è uno dei criteri importanti da
prendere in considerazione per migliorare il livello di autonomia del robot.
Nella pianificazione del percorso, la sicurezza è un problema importante che
dovrebbe essere preso in considerazione al fine di garantire che un robot
raggiunga la posizione target senza collisioni con gli ostacoli circostanti.
Inoltre, ci sono aspetti importanti che devono essere affrontati nella
pianificazione del percorso; tempo computazionale, percorso ottimale e
completezza.
Uno dei metodi più diffusi per la pianificazione dei percorsi è il metodo
\emph{Campi Potenziali Artificiali}.
Il metodo del potenziale è in grado di superare uno scenario sconosciuto,
tenendo conto della realtà dell'ambiente corrente e del movimento del robot.
Due tipi di forze sono coinvolte nel metodo del campo potenziale;
forza attrattiva generata da obiettivi e forza repulsiva generata da ostacoli;
dalla differenza dei due campi il robot deve riprogrammare un nuovo
percorso\cite{apotetianfield}.
Utilizzando informazioni parziali sullo spazio di lavoro raccolte attraverso i
sensori, quindi le informazioni sensoriali sono integrate in una mappa secondo
un paradigma \emph{sense---plan---move}.
Oppure utilizzare le informazioni sensoriali impiegate per pianificare moti
secondo un paradigma \emph{stimulus---response} (navigazione reattiva).
Il robot è considerato come un punto materiale sotto l'influenza dei campi
prodotti da obiettivi e ostacoli nello spazio di ricerca. Le forze repulsive
sono generate da ostacoli mentre la forza attrattiva è generata dagli obiettivi.
La forza risultante (la somma di tutte le forze) dei campi sul robot viene
utilizzato per determinare la direzione del movimento e la velocità di
spostamento evitando collisioni\cite{5498220}.
Tuttavia esistono svantaggi quali:
\begin{enumerate*}[label={\alph*)},font={\bfseries}]
\item situazione di stallo dovuta ai minimi locali;
\item oscillazione in presenza di ostacoli;
\item nessun passaggio tra ostacoli ravvicinati;
\item oscillazioni in passaggi stretti\cite{131810}.
\end{enumerate*}
Il robot viene considerato come punto $\mathbf{q} = (x \, y)^T$, in un piano
cartesiano, attratto (potenziale $U_{\text{att}}$) dal punto obiettivo
$\mathbf{q}_g$ e respinto (potenziale $U_{\text{rep}}$) dagli ostacoli.
\begin{equation}
\label{eq:apfm}
U(q) = U_{\text{att}}(q) + U_{\text{rep}}(q)
\end{equation}
%
\noindent dove $U(q)$ potenziale artificiale; $U_{\text{att}}(q)$ campo
attrattivo; $U_{\text{rep}}(q)$ campo repulsivo.
La pianificazione avviene in modo incrementale: ad ogni configurazione
$\mathbf{q}$, la forza artificiale viene generata come nell'equazione (\ref{eq:attractive}):
\begin{equation}
\label{eq:attractive}
\begin{split}
F(q) &= - \nabla U(q)\\
&= - \nabla U_{\text{att}}(q) -U_{\text{rep}}(q)\\
F(q) &= F_{\text{att}}(q) + F_{\text{rep}}(q)
\end{split}
\end{equation}
dove $F(q)$: forza artificiale; $F_{\text{att}}(q)$: forza attrattiva;
$F_{\text{rep}}(q)$: forza repulsiva. Il campo potenziale $U_{\text{att}}$) tra
robot e obiettivo viene descritto da (\ref{eq:fieldattractive}) per trascinare
il robot nell'area obiettivo.
\begin{equation}
\label{eq:fieldattractive}
\begin{split}
U_{\text{att}}(q) &= \frac{1}{2} \, k_a \, (q-q_d)^2\\
&= \frac{1}{2} \, k_a \, \rho^{2}_{goal}(q)
\end{split}
\end{equation}
dove $k_a$: coefficiente positivo per APF\footnote{Artificial Potential Field};
$q$: posizione corrente del robot; $q_{d}$: posizione corrente dell'obiettivo.
\begin{equation}
  \rho_{\text{goal}}(q) = \|q-q_{d}\|
\end{equation}
è una distanza euclidea dalla posizione del robot alla posizione dell'obiettivo.
La forza attrattiva del robot è calcolata come gradiente negativo del potenziale
campo\cite{6283526}:
\begin{equation}
\label{eq:gradientfield}
\begin{split}
F_{\text{att}}(q) &= -\frac{1}{2}k_a \rho^2_{\text{goal}}(q)\\
F_{\text{att}}(q) &= -k_a \, (q-q_d)
\end{split}
\end{equation}
%
$F_{\text{att}}(q)$, nell'eq. (\ref{eq:gradientfield}), è un vettore diretto
verso $q_{\text{d}}$ con intensità linearmente proporzionale alla distanza da
$q$ a $q_{\text{d}}$. Può essere scritto nelle sue componenti:
\begin{equation}
\label{eq:componet}
\begin{split}
F_{\text{att}} -x(q) &= -k_a \, (x - x_d)\\
F_{\text{att}} -y(q) &= -k_a \, (y - y_d)\\
\end{split}
\end{equation}
%
Le equazioni (\ref{eq:componet}) sono la forza attrattiva nelle direzioni $x$
e $y$. Nella funzione potenziale, il robot deve essere respinto dagli ostacoli,
ma se lontano da questi, il movimento non risente della loro influenza.
La funzione potenziale di repulsione (\ref{eq:repulsive}) è:
\begin{equation}
\label{eq:repulsive}
U_{\text{rep}}(q) =
\begin{cases}
\frac{1}{2}k_b(\frac{1}{d(q)}-\frac{1}{d_0})^2 &\mbox{se } d(q) \leq d_0 \\
0 & \mbox{se } d(q) > d_0
\end{cases}
\end{equation}
\begin{figure}[!h]
\centering
\subfloat[][\emph{Campo attrattivo - repulsivo}.]
   {\includegraphics[width=.35\textwidth]{image008}} \,
\subfloat[][\emph{Pianificazione traiettoria}.]
   {\includegraphics[width=.35\textwidth]{image010}}
\caption{Potenziali artificiali}
\label{fig:potentialfield}
\end{figure}


