La programmazione orientata agli oggetti è un paradigma di
programmazione che permette di definire oggetti software in grado di interagire
gli uni con gli altri attraverso lo scambio di messaggi. È particolarmente
adatta nei contesti in cui si possono definire delle relazioni di interdipendenza
tra i concetti da modellare.
Tra gli altri vantaggi della programmazione orientata agli oggetti:
\begin{enumerate*}[label={\alph*)},font={\bfseries}]
\item fornisce un supporto naturale alla modellazione software degli oggetti del
mondo reale o del modello astratto da riprodurre;
\item permette una più facile gestione e manutenzione di progetti di grandi
dimensioni;
\item l'organizzazione del codice sotto forma di classi favorisce la modularità
e il riuso di codice.
\end{enumerate*}
% Le classi sono uno strumento per costruire strutture dati che contengano non
% solo dati ma anche il codice per gestirli.
% Come tutti i costrutti che permettono di definire le strutture dati, una classe
% definisce un nuovo tipo di dato.
% I membri di una classe sono dati, chiamati attributi, e metodi, ovvero
% procedure, che operano su un oggetto.
% Dal punto di vista matematico, una classe definisce un insieme in modo
% intensivo, ovvero definendone le caratteristiche invece che elencandone gli
% elementi.
% Se l'accesso agli attributi è ristretto ai soli membri della classe,
% le caratteristiche dell'insieme possono includere vincoli sui possibili valori
% che la tupla degli attributi può o non può assumere, e anche sulle possibili
% transizioni tra questi stati. Un oggetto può quindi essere visto come una
% macchina a stati finiti.
% Una classe può dichiarare riservate una parte delle sue proprietà e/o dei suoi
% metodi, e riservarne l'uso a sé stesso e/o a particolari tipi di oggetti a lui
% correlati.
% Un oggetto è una istanza di una classe. Un oggetto occupa memoria, la sua
% classe definisce come sono organizzati i dati in questa memoria.
% Ogni oggetto possiede tutti gli attributi definiti nella classe, ed essi hanno
% un valore, che può mutare durante l'esecuzione del programma come quello di
% qualsiasi variabile.
Il paradigma OOP suggerisce un principio noto come \emph{information hiding} che
indica che si debba accedere agli attributi dell'istanza solo tramite metodi
invocati su quello stesso oggetto.
% Sintatticamente, i metodi di una classe vengono invocati ``su" un particolare
% oggetto, e ricevono come parametro implicito l'oggetto su cui sono stati
% invocati.
L'incapsulamento è la proprietà per cui i dati che definiscono lo stato interno
di un oggetto e i metodi che ne definiscono la logica sono accessibili ai metodi
dell'oggetto stesso, mentre non sono visibili ai client. Per alterare lo stato
interno dell'oggetto, è necessario invocarne i metodi pubblici, ed è questo lo
scopo principale dell'incapsulamento. Infatti, se gestito opportunamente, esso
permette di vedere l'oggetto come una black-box, con la quale l'interazione
avviene solo e solamente tramite i metodi definiti dall'interfaccia.
Il punto è dare delle funzionalità agli utenti nascondendo i dettagli legati
alla loro implementazione.\cite{cox1991object}
