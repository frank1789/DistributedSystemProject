\section{Conclusioni}
\label{sec:conclusioni}
È stato qui presentato un modo efficiente di condurre un'esplorazione coordinata
di un ambiente.
Possibile passo in avanti per la suddetta applicazione sarebbe la sua
realizzazione ed esecuzione fisica in modo da condurre uno studio più
approfondito sulle possibili problematiche riscontrabili.
Un limite evidente è la pianificazione di traiettorie come visto negli svantaggi
del metodo dei potenziali artificiali, nella sezione
(\ref{sec:soluzioneprop}.\ref{ssec:ArtPotField}), poiché viene perso molto tempo
nel modificare la traiettoria specialmente in prossimità degli ostacoli,
in sezioni di passaggio, e in ambienti ciechi.
Attualmente il robot si muove con velocità costante ponendo un freno allo spazio
esplorabile, un modello cinematico migliore permetterebbe lo studio del cambio
di velocità e accelerazione adattandole quando è necesario affrontare
deviazioni e in presenza o meno di ostacoli.
Inoltre si evidenzia che la generazione procedurale di mappe, vista nella
sezione (\ref{sec:soluzioneprop}.\ref{ssec:generazioneproc}), permette una
generazione di casistiche ampia, ma poco controllabile e incapace di riprodurre
strutture realmente esistenti nell'ambiente reale.
Una possibile evoluzione sarebbe la ricostruzione di uno scenario 3D.
Altro limite attuale dell'applicazione è il non riconoscimento di zone
inaccessibili, ciò comporta nel caso di zone inaccessibili molto estese una gran
probabilità di assegnazione di un target appartenente a tale zona per il robot.
Altri esempi di possibili migliorie software apportabili,sono: una migliore
rappresentazione fisica della comunicazione, robot-robot e robot-ancora
\textsc{Wi-Fi}, tramite gli ultimi protocolli esistenti e l'adozione di algoritmi 
per una stima distribuita della posizione (least-square o di kalman distribuito).

