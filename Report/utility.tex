\subsection{Allocazione destinazioni ed Utility}
\label{ssec:utilty}
L'obiettivo di un processo di esplorazione è quello di coprire l'intero
ambiente nel minor tempo possibile. Pertanto, è essenziale che i robot
tengano traccia di quali aree dell'ambiente sono già state esplorate.
Inoltre, i robot devono costruire una mappa globale per pianificare i loro
percorsi e coordinare le loro azioni.
Quando esploriamo un ambiente sconosciuto, siamo particolarmente
interessati alle ``celle di frontiera".
Con cella di frontiera, denotiamo ciascuna cella già esplorata che è un vicino
immediato di un cella sconosciuta, inesplorata.
Se indirizziamo un robot a una tale cella, ci si può aspettare che ottenga
informazioni sull'area inesplorata quando arriva alla sua posizione di
destinazione.
Il fatto che una mappa in generale contiene diverse aree inesplorate solleva
il problema di come assegnare in modo efficiente le destinazioni a più robot.
Se sono coinvolti più robot, vogliamo evitare che molti di loro si spostino
nella stessa posizione. Per affrontare questi problemi e determinare i luoghi
di destinazione appropriati per i singoli robot, il nostro sistema utilizza il
seguente approccio: considera contemporaneamente il costo del raggiungimento di
una cella di frontiera e l'utilità di questa.
Per ogni robot, il costo per raggiungere la cella è proporzionale alla distanza
tra se stesso e quest'ultima, d'altra parte l'utilità di una cella di frontiera
dipende invece dal numero di robot che si stanno spostando in quella direzione o
in posizioni limitrofe.\cite{burgard2005coordinated}
Di seguito vengono riportate le equazioni implementate:
\begin{equation}
\label{eq:componet1}
U(t_n|t_1,\dots,t_{n-1} ) = U_{t_n} -  \sum_{i=1}^{n-1} P(||t_n - t_i ||)
\end{equation}

\begin{equation}
\label{eq:componet2}
P(d) = \begin{cases}
   1 - \frac{d}{maxrange}  	&  	\mbox{se } d < \text{maxrange} \\
   0 									& 		\text{altrimenti}
   \end{cases}
\end{equation}

\begin{equation}
\label{eq:componet3}
(i,t) = \text{argmax}_{i^{\iota},t^{\iota}} (U_{t^{\iota}}  -\beta V^{i^{\iota}}_{t^{\iota}} )
\end{equation}
%
La prime due equazioni (\ref{eq:componet1}--\ref{eq:componet2}), calcolano
l'utility per il robot n-esimo mentre l'ultima equazione rappresenta il caso
esteso all'integrazione di più robot che comunicano tra loro.
\emph{Maxrange} rappresenta la massima visibilità fornita dal \textsc{lidar},
$d$ rappresenta la distanza dei target assegnati ai vari robot, $U$ la funzione
di utility e $V$ la funzione di costo, quest'ultima tiene conto della distanza
di ogni robot dal suo target. 
Le $i$ e $t$, nell'eq. \eqref{eq:componet3}, forniscono per ogni robot il suo
target i-esimo ottimale.
Il modo in cui questi target da parte di ogni robot vengono raggiunti è
lasciato alla funzione potenziale che verrà descritta in seguito.
