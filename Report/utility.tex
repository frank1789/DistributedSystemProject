\subsection{Allocazione destinazioni ed Utility}
L'obiettivo di un processo di esplorazione è quello di coprire l'intero ambiente nel minor tempo possibile. Pertanto, è essenziale che i robot tengono traccia di quali aree dell'ambiente sono già state esplorate. Inoltre, i robot devono costruire una mappa globale per pianificare i loro percorsi e coordinare le loro azioni.

Quando esploriamo un ambiente sconosciuto, siamo particolarmente interessato alle "celle di frontiera". 
Con cella di frontiera, denotiamo ciascuna cella già esplorata che è un vicino immediato di un cella sconosciuta, inesplorata. 
Se indirizziamo un robot a una tale cella, ci si può aspettare che ottenga informazioni sull'area inesplorata
quando arriva alla sua posizione di destinazione.
Il fatto che una mappa in generale contiene diverse aree inesplorate solleva il problema di come assegnare compiti di esplorazione rappresentati da celle di frontiera ai robot.
Se sono coinvolti più robot, vogliamo evitare che molti di loro si spostino nella stessa posizione. Per affrontare questi problemi e determinare i luoghi di destinazione appropriati per il i singoli robot il nostro sistema utilizza il seguente approccio: consideriamo contemporaneamente il costo del raggiungimento di una cella di frontiera e l'utilità di quella cella.
Per ogni robot, il costo di una cella è proporzionale alla distanza tra il robot e quella cella e l'utilità di una cella di frontiera dipende invece dal numero di robot che si stanno spostando in quella cella o in un posto vicino a quella cella.
Di seguito vengono riportate le equazioni implementate
\begin{equation}	
\label{eq:componet}
U(t_n|t_1,....,t_{n-1} ) = U_{t_n} -  \sum_{i=1}^{n-1} P(||t_n - t_i ||)
\end{equation}

\begin{equation}	
\label{eq:componet}
P(d) = \begin{cases} 
   1 - \frac{d}{max_range}   \quad \quad if d < max_range \\  0 \quad \quad otherwise
   \end{cases}
\end{equation}

\begin{equation}	
\label{eq:componet}
(i,t) = argmax_{i^{\iota},t^{\iota}} (U_{t^{\iota}}  -\beta V^{i^{\iota}}_{t^{\iota}} ) 
\end{equation}

La prime due equazioni rappresenta come viene calcolata la l'utility per il robot n-esimo mentre l'ultima equazione rappresenta il caso allargato all'integrazione di più robot che comunichino tra loro. Maxrange rappresenta la massima visibilità fornita dal lidar, d rappresenta la distanza dei target assegnati ai vari robot la U la funzione di utility e la V la funzione di costo che tiene conto della distanza di ogni robot dal suo target. Le i e t risultanti dall'ultima equazione forniscono per ogni robot il suo target i-esimo ottimo per avere l'ottimo globale della funzione tra le parentesi. 
Il modo in cui questi target da parte di ogni robot vengono raggiunti è lasciato alla funzione potenziale che verrà descritta in seguito.