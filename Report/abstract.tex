\begin{abstract}
\noindent In questo lavoro, consideriamo il problema di esplorare un ambiente
sconosciuto con un team di robot.
Come nell'esplorazione di robot singoli, l'obiettivo è di ridurre al minimo il
tempo di esplorazione complessivo.
Il problema chiave da risolvere nel contesto di robot multipli è quello di
scegliere i punti di destinazione appropriati per i singoli robot in modo che
possano esplorare contemporaneamente diverse regioni dell'ambiente. Presentiamo
un approccio per il coordinamento di più robot, che tiene conto simultaneamente
del costo di raggiungere un punto target e della sua utilità. 
Descriviamo inoltre come il nostro algoritmo può essere esteso a situazioni
in cui il raggio di comunicazione dei robot è limitato.
Per la stima delle posizioni dei robot è stato utilizzato il filtro
particellare, assumendo una comunicazione con delle ancore \textsc{wi-fi}.
I risultati dimostrano che la nostra tecnica distribuisce efficacemente i robot
sull'ambiente e consente loro di compiere rapidamente la loro missione.
\end{abstract}
