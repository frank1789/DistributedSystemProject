\section{Introduzione}
\lettrine[lines=3]{L'}{esplorazione} efficiente di ambienti sconosciuti è un 
problema fondamentale nella robotica mobile. 
L'estensione alla esplorazione di più robot pone diverse nuove sfide, tra cui: 
coordinamento di robot, integrazione delle informazioni raccolte dai robot in
una mappa coerente; e comunicazione limitata.
Il coordinamento di un sistema di più robot è la base per un'efficiente 
implementazione di una esplorazione distribuita. Difficoltà maggiore nel campo 
del coordinamento vengono dalle conoscenze che si assumono avere i robot sul
comportamento degli altri robot. 
Se i robot conoscono le loro posizioni relative e condividono una mappa della 
zona che hanno esplorato finora, si può raggiungere un coordinamento efficace,
guidando i robot in zone inesplorate dell'ambiente.
Questo può essere fatto assegnando ai robot il compito di raggiungere un loro 
punto di arrivo preso dalla frontiera fornita dalla scansione del 
\emph{\textsc{lidar}}\footnote{LIDAR (acronimo dall'inglese Light Detection and 
Ranging o Laser Imaging Detection and Ranging) è una tecnica di 
telerilevamento che permette di determinare la distanza di un oggetto o di una 
superficie utilizzando un impulso laser, ma è anche in grado di determinare la
concentrazione di specie chimiche nell'atmosfera e nelle distese d'acqua.}.\cite{131810}
Per un assegnamento efficace è importante che i robot quando vadano a 
condividere dei punti comuni all'interno delle loro frontiere che non cerchino 
di raggiungere lo stesso punto.
Per la stima della posizione viene adottato un filtro particellare, i robot 
comunicando con delle ancore \textsc{wi-fi} riescono a conoscere con un dato 
errore la loro posizione ricoperta in quell'istante. Per il raggiungimento 
della zona target prefissata viene invece adottata la funzione potenziale.
I robot contengono al loro interno memoria delle zone già esplorate e non, 
ipotizzando di conoscere in anticipo le dimensioni massime della mappa che 
andranno ad esplorare quest'ultimi costruiscono attraverso l'integrazione di 
successive matrici di occupazione locale una matrice di occupazione globale 
che rappresenta la mappatura dell'ambiente.
Saranno successivamente presentati la modellazione del sistema e i successivi 
risultati ottenuti.
